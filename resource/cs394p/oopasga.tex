\documentstyle[12pt]{article}
\setlength{\oddsidemargin}{0 in}
\setlength{\textwidth}{6.5 in}
\setlength{\textheight}{9.0 in}
\setlength{\parskip}{0.1 in}
\setlength{\topmargin}{-0.4in}
\begin{document}
\begin{center} {\Large {\bf Object-oriented Programming: Examples}}
\end{center}

\vspace{.15 in}

\section{Introduction}

This exercise introduces a simple object-oriented programming system,
{\tt OOPS}, which is patterned after CLOS (Common Lisp Object Standard),
using explicit message sending rather than generic functions.  Several
exercises are given to illustrate the use of the {\tt OOPS} system and
to illustrate interesting features of object-oriented programming.
A subsequent exercise will be for the student to implement the {\tt OOPS}
system itself in Lisp.

\section{The OOPS System}

The {\tt OOPS} system is provided in the file {\tt oops.b}.  The individual
functions provided by {\tt OOPS} are described below.

\begin{verbatim}

(defclass <class-name> (<superclass-name>*)
                       (<slot-spec>*)
                       <class-options>*)

\end{verbatim}
{\tt defclass} is the function that defines a class.  {\tt <class-name>} is the
name of the class.  The second argument is a list of the names of superclasses
of the class, and the third argument is a list of slot specifications.
The {\tt <class-options>} are specifications that apply to the class as a
whole, including documentation.  The arguments of {\tt defclass} are not
evaluated.

A {\tt <slot-spec>} describes a slot that will appear in instances of the
class. It is a list beginning with the name of the slot and followed by slot
options.  Each option is a pair consisting of a keyword for the option
followed by a value.  The option keywords that are defined are:
\begin{verbatim}

:initarg   <symbol>

:initform  <form>

\end{verbatim}
The {\tt :initarg} option specifies a name by which the slot can be initialized
when making a new instance with {\tt make-instance}; often, it is the name
of the slot prefixed by a colon.  The {\tt :initform} option specifies a
form that is evaluated to initialize the slot when no {\tt :initarg}
initialization is specified by the user.  The {\tt <form>} is evaluated
each time a new instance is created.

An example of a call to {\tt defclass} is:
\begin{verbatim}
(defclass xyvector (vector)
  ((x :initarg :x :initform 0)
   (y :initarg :y :initform 0))
  (:documentation "A simple x-y vector"))

\end{verbatim}

\begin{verbatim}
(classp <class>)

(class-of <instance>)

(describe <instance>)

\end{verbatim}
{\tt classp} tests whether its argument is a class.  {\tt class-of} returns
the class of its argument, or the type of basic Lisp objects.  {\tt describe}
prints a simple description of the object given as its argument.

\begin{verbatim}
(make-instance <class> <init>*)

\end{verbatim}
{\tt make-instance} makes a new instance object of the class given by
{\tt <class>}.  Slot values of the new instance may be initialized by
specifying the {\tt :initarg} names as shown in the class definition,
followed by corresponding values.  For example:
\begin{verbatim}

(make-instance 'xyvector :x 3 :y 4)
\end{verbatim}

\vspace{.3in}

\begin{verbatim}
(defmethod <selector> <args>) <code>)
\end{verbatim}

{\tt defmethod} is used to define a {\it method}, that is, a function that
implements a message for a given kind of object.  {\tt <selector>} is the name
of the message.  The list of arguments is similar to the argument list for
a Lisp function, except that instead of being a simple variable name, an
argument may have the form {\tt (<var> <type>)} where {\tt <var>} is a
variable name and {\tt <type>} is the class of the argument.  The first
argument used with {\tt defmethod} {\it must} be of this form; the method
is stored with the class of the first argument.  For example:
\begin{verbatim}

(defmethod area ((c circle))
  (* 3.1415926 (expt (slot-value c 'radius) 2)))
\end{verbatim}

\vspace{.3in}

\begin{verbatim}
(slot-value <object> <slot-name>)
\end{verbatim}
{\tt slot-value} retrieves the value stored in the slot {\tt <slot-name>} of
the instance {\tt <object>}.  {\tt slot-value} can also be used with {\tt setf}
to store slot values.  For example:
\begin{verbatim}

(setf (slot-value c 'radius) 7.0)
\end{verbatim}

\vspace{.3in}

\begin{verbatim}
(sendm <object> <selector> <args>)
\end{verbatim}
{\tt sendm} is the message-sending function.  It sends to {\tt <object>} the
message whose {\tt <selector>} is specified, with optional arguments
{\tt <args>}.
For example, to get the area of a circle {\tt c}, we could use:
\begin{verbatim}

(sendm c 'area)
\end{verbatim}
The use of {\tt sendm} differs from CLOS, but is basically doing the same
thing that CLOS does.  In CLOS, sending a message is performed by the generic
function mechanism, so that the CLOS form corresponding to our {\tt sendm}
call is:
\begin{verbatim}

(<selector> <object> <args>)
\end{verbatim}
For the example of finding the area of circle {\tt c}, in CLOS one would use:

\begin{verbatim}

(area c)
\end{verbatim}

\section{Vector Example}

The file {\tt oopexa.lsp} contains some simple class definitions and methods
for vectors.  This example illustrates how objects that are implemented in
very different ways can look the same through their external message
interfaces.  This file contains definitions of an {\tt xyvector} and an
{\tt rthvector}.  The generic method {\tt +} for vectors will add two vectors,
producing a new vector whose type is the same as the type of its first
argument.

The file {\tt oopexa.tst} contains some test cases to illustrate operations
on vectors.  Step through this test file to see the effects of the commands
that are given.

The existing methods make an {\tt rthvector} look like an {\tt xyvector}
externally.  Write methods to make an {\tt xyvector} usable as if it
were an {\tt rthvector}.  Test your new methods.

\section{Planet Example}

This example illustrates how object-oriented programming can
provide generality by allowing a method to work for objects whose
internal structures are quite different; it also illustrates the use of
multiple inheritance.  The example shows how a
method that defines the {\tt density} of a physical object can work
for objects as diverse as planets, bricks, and bowling balls.

Write class definitions for the classes {\tt physical-object}, {\tt sphere},
{\tt parallelepiped}, {\tt planet}, {\tt ordinary-object}, {\tt brick}, and
{\tt bowling-ball}.  {\tt physical-object} has no slots and no supers, but
it has a method {\tt density}, which is defined as {\tt mass} divided by
{\tt volume}.  {\tt sphere} has no supers or slots; it has
a method to compute {\tt volume} ($ 4/3 * \pi * radius^{\hbox{3}} $).
{\tt parallelepiped} has no supers or slots; it has a method to compute
{\tt volume} from {\tt length}, {\tt width}, and {\tt height}.  An
{\tt ordinary-object} has as a superclass {\tt physical-object}; it has a
method to compute {\tt mass} as {\tt weight}\ /\ 9.88 .  (We will assume
MKS units.)\  A {\tt planet} is a {\tt physical-object} and a {\tt sphere};
it has a {\tt mass} slot and a {\tt radius} slot.  A {\tt brick} is an
{\tt ordinary-object} and a {\tt parallelepiped}; it has slots {\tt weight},
{\tt length}, {\tt width}, and {\tt height}.  A {\tt bowling-ball}
is an {\tt ordinary-object} and a {\tt sphere}; it has a {\tt type} slot,
which has the possible values {\tt adult} or {\tt child}, and a method to
compute its {\tt weight} (we will assume a weight of 8 kg for an adult ball
and 4 kg for a child's ball; we will assume that the {\tt radius} of both
types is 0.1 meter).

First write the class definitions to describe this taxonomy of objects.
Next, make some instance objects to use as test data.  (The earth has a
radius of 6.37E6 meters and a mass of 5.98E24 kg.)\  Your ultimate goal
is to be able to find the {\tt density} of a planet, brick, or bowling-ball
using the single method defined under {\tt physical-object}.

\section{Iterators}

An {\it iterator} is an object that steps through a sequence.  It has a set
of parameters and a state.  It accepts a message {\tt initialize}, which
causes it to initialize its state from its parameters, and a message
{\tt next}, which causes it to return the next item in the sequence and
update its state.  After the sequence is finished, {\tt next} returns the
value {\tt NIL}.

Define objects for the following iterators:
\begin{enumerate}
\item A {\tt for-loop} iterator, with parameters {\tt start}, {\tt end}, and
{\tt step}, and internal state {\tt current}.

\item A {\tt list-iterator} that steps through the items in a list.

\item A {\tt filter-iterator} that takes another iterator and a predicate, and
returns the items from the other iterator that pass the predicate test.

\item A {\tt prime-generator} that generates primes by filtering a sequence.
\end{enumerate}

\section{Virtual Files}

A file can be considered to be a sequence of lines.  A file object has
a message {\tt initialize}, which causes it to initialize its state from
its parameters, and a message {\tt next}, which causes it to return the
next line in the file and update its state.  After the file is finished,
{\tt next} returns the value {\tt NIL}.  A file object has a {\tt file-name}
slot, and perhaps others.

It would be nice to have virtual files such as the following:
\begin{enumerate}
\item A {\tt file} object that returns each line in a file.

\item A {\tt partial-file} object that has a {\tt line-numbers} parameter
(a list of pairs of line numbers, in ascending order).  This object
returns only those lines specified in its line-number list.

\item A {\tt concatenated-file} object that concatenates the results from a
list of other file iterators.
\end{enumerate}

\end{document}
